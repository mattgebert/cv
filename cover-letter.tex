\documentclass[10pt,a4paper,ragged2e,withhyper,paragraphstrue]{altacv}

% \newenvironment{sloppypar*}{\sloppy\ignorespaces}{\par}
\geometry{left=1.2cm,right=1.2cm,top=1cm,bottom=1cm,columnsep=0.75cm}
\usepackage{subfiles} % Use subfiles to define personal information.
\usepackage{paracol}

% % Change the font if you want to, depending on whether
% % you're using pdflatex or xelatex/lualatex
% \ifxetexorluatex
%   % If using xelatex or lualatex:
%   \setmainfont{Roboto Slab}
%   \setsansfont{Lato}
%   \renewcommand{\familydefault}{\sfdefault}
% \else
%   % If using pdflatex:
%   \usepackage[rm]{roboto}
%   \usepackage[defaultsans]{lato}
%   % \usepackage{sourcesanspro}
%   \renewcommand{\familydefault}{\sfdefault}
% \fi

% Fork (before v1.6.5a): Change the color codes to test your personal variant on any mode
\ifdarkmode%
  \definecolor{PrimaryColor}{HTML}{C69749}
  \definecolor{SecondaryColor}{HTML}{D49B54}
  \definecolor{ThirdColor}{HTML}{1877E8}
  \definecolor{BodyColor}{HTML}{ABABAB}
  \definecolor{EmphasisColor}{HTML}{ABABAB}
  \definecolor{BackgroundColor}{HTML}{191919}
\else%
%  \definecolor{PrimaryColor}{HTML}{001F5A}
%  \definecolor{SecondaryColor}{HTML}{0039AC}
%  \definecolor{ThirdColor}{HTML}{F3890B}
%  \definecolor{BodyColor}{HTML}{666666}
%  \definecolor{EmphasisColor}{HTML}{2E2E2E}
%  \definecolor{BackgroundColor}{HTML}{E2E2E2}
  \definecolor{PrimaryColor}{HTML}{001F5A}
  \definecolor{SecondaryColor}{HTML}{448888}
  \definecolor{ThirdColor}{HTML}{4682B4}
  \definecolor{BodyColor}{HTML}{666666}
  \definecolor{EmphasisColor}{HTML}{2E2E2E}
  \definecolor{BackgroundColor}{HTML}{FFFFFF}
%  \definecolor{BackgroundColor}{HTML}{E2E2E2}
\fi%

%My Colours:
% Teal 448888
% Orange DD5522 
% Light blu 77EEEE
% Steel Blu 4682B4

\colorlet{name}{PrimaryColor}
\colorlet{tagline}{SecondaryColor}
\colorlet{heading}{PrimaryColor}
\colorlet{headingrule}{ThirdColor}
\colorlet{subheading}{SecondaryColor}
\colorlet{accent}{SecondaryColor}
\colorlet{emphasis}{EmphasisColor}
\colorlet{body}{BodyColor}
\pagecolor{BackgroundColor}

%\usepackage{hyperref
%%Setup bib hyperlinks
\hypersetup{%
	urlcolor=SecondaryColor,
	citecolor=SecondaryColor
}

% Change some fonts, if necessary
\renewcommand{\namefont}{\Huge\rmfamily\bfseries}
\renewcommand{\personalinfofont}{\small\bfseries}
\renewcommand{\cvsectionfont}{\LARGE\rmfamily\bfseries}
\renewcommand{\cvsubsectionfont}{\large\bfseries}

% Change the bullets for itemize and rating marker
% for \cvskill if you want to
\renewcommand{\itemmarker}{{\small\textbullet}}
\renewcommand{\ratingmarker}{\faCircle}

\newcommand{\textalignment}{
	\justifying
    \tolerance=1 %https://tex.stackexchange.com/a/5039/137109
    \emergencystretch=\maxdimen
    \hyphenpenalty=10000 
    \hbadness=10000
}

\setlength{\parindent}{24pt}
\newcommand{\pind}{\hspace{24pt}}

\begin{document}
    
    \IfFileExists{./cl-intro.info}{
            \subfile{./cl-intro.info}
        }{
            \subfile{./cl-intro-example.info}
        }

    \vspace{1em}
	% Proposal describing how attending the 5th AOFSRR school will hlep facilitate my synchrotron-based research program.
	% Structure:
	% 1. Who I am
	% 2. What I have achieved
	% 3. What I intend to do
	% 4. Why 5th AOFSRR is essential to what I intend to do 
	% 5. What I hope to contribute in the future.
	% 6. Summary why I am a valuable member to have

    {\color{emphasis}
    \textalignment
    \pind I am writing to you to apply for an invitation to the\cvreference{5\textsuperscript{th} AOFSRR school.}{https://events01.synchrotron.org.au/event/183/overview}
    % 1. Who I am
    As of February, I have been conferred for my doctorate in condensed-matter physics, undertaken in the ARC Centre of Excellence in Future Low Energy Electronics and Technologies (FLEET), under the supervision of FLEET Director Prof. Michael Fuhrer.
    I have recently started a postdoctoral research position under\cvreference{Prof. Christopher McNeill,}{https://www.monash.edu/engineering/chrismcneill}who specialises in using synchrotron light to understand the properties of organic semiconductors. Thus the timing for this school is impeccable for helping me pursue and understand theory and applications of synchrotron radiation.
    
    % 2. What I have achieved
  	\pind During my doctoral research, I investigated Dirac materials such as graphene and topological insulators. These novel materials exhibit different electronic behaviour due to the Dirac surface states that arise from topology and symmetry. The topological features they possess can be used to reduce back-scattering, preserving excellent conductive properties even in the presence of disorder. These materials require significant fab processing to perform cryogenic and electronic measurements; I have extensive cleanroom and synthesis experience across a variety of lithographic, deposition and characterization techniques. 
  	One research highlight is\cvreference{10.1021/ACS.NanoLett.2c03492,}{https://pubs.acs.org/doi/full/10.1021/acs.nanolett.2c03492}where I investigated the influence of a novel liquid-metal printed dielectric on the Dirac surface states of graphene, with liquid-metal collaborators at RMIT.  	
  	Additionally I've performed neutron and X-ray reflectometry techniques at ANSTO's Australian Centre for Neutron Scattering (ACNS), in collaboration with Dr. David Cortie (ACNS) and Dr. Golrokh Akhgar (CSIRO). We did this to probe the interaction between magnetic insulators and the surface states of the topological features, and have a manuscript in preparation.
  	    
    % 3. What I intend to do
    \pind In my new role with Prof. McNeill, I will focus on harnessing resonant tender X-rays to understand the microstructure of organic semiconductor films. There are multiple reasons tender X-rays can be used to provide structural precision, including imaging resolution due to wavelength, relevant elements with absorption edges used in organic semiconductors, and the decreased absorption by the environment. By using the absorption edge, pre- \& post-edge contrast can be leveraged to understand the elemental contribution. However, this technique is complex in that there are many signal contributions from geometry, polarisation, diffraction, reflection e.t.c., that need to be detangled. The intention of this project is to demonstrate the successful use of technique and develop significant analysis software architecture to equip additional research. 
    Successfully utilising tender x-rays will not only offer strong information into the packing of such polymers, but also allow liquid cell measurements of such suspended polymers, where such apparatuses in soft-Xray regimes would have suffered from absorption due to solvents - tender edges provide an exclusive probe to the polymer without solvent interference. 
    
    % 4. Why 5th AOFSRR is essential to what I intend to do 
    \pind The 5\textsuperscript{th} AOFSRR School is essential for me to gain expert knowledge in a variety of synchrotron techniques and capabilities. The combination of lectures and practical sessions will fill in gaps of understanding as well as provide me general awareness of common best-practice for such research. The opportunity to extend my professional network, not only with the staff at the Australian Synchrotron, but with a group of users within Asia-Oceania will be immensely valuable. The opportunity to conduct a practical on two beamlines will also offer insight into their advanced operation, particularly ones that I need to use, but am inexperienced in. For example, I recently performed my first GI-WAXS in April, but I am not familiar alternative modes of operation for the SAXS-WAXS beamline and the implication of detector distances. MEX and SXR are also very important beamlines for my main project. I'd also be fascinated to see the infrared Microscopy (IRM) beamline.
    
    % 5. What I hope to contribute in the future.
    \pind I have strong expectations that my future career pathways will be contributing to synchrotron science, either as a researcher, instrument scientist or controls/algorithmic contributor. I have already begun contributing to a new NEXAFS anaylsis repository, and will be developing further tools throughout my postdoctoral work in conjunction with beamtimes at the Australian Synchrotron. My diverse background (with a double undergraduate degree in Electrical and Computer systems, in addition to science) means I can communicate across a significant number of stakeholder groups within synchrotron research, which makes me an exceptional candidate to be part of a program like this.
    
    % 6. Summary why I am a valuable member to have
    \pind As I have indicated, I believe this school is of real value to my development as I have begun to gain expertise in Synchrotron science. I have clear and long-term commitment to contributing to diverse and complex science, and all aspects of my skillset (and it's breadth) can be utilised in a synchrotron context, making me an excellent choice of candidate to award an Australian placement in the AOFSRR school. I am very excited by the opportunity to be a part of such a program. 
    
    \pind If you have any questions or seek further clarification, please don't hesitate to contact me. I look forward to hearing from you soon.
    
    \vspace{1em}

    Yours Sincerely, \newline

    {\color{emphasis}Matthew G Gebert}

    }
    
    % \vspace*{0.5em}
    \divider
    \IfFileExists{./cl-personal.info}{
        \vspace{-2em}
        \subfile{./cl-personal.info}
    }{
        \vspace{-2em}
        \subfile{./cl-personal-example.info}
    }

\end{document}