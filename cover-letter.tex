\documentclass[10pt,a4paper,ragged2e,withhyper,paragraphstrue]{altacv}

% \newenvironment{sloppypar*}{\sloppy\ignorespaces}{\par}
% \geometry{left=1.2cm,right=1.2cm,top=1cm,bottom=1cm,columnsep=0.75cm}
% \geometry{left=2.54cm,right=2.54cm,top=2.54cm,bottom=2.54cm,columnsep=0.75cm}
\geometry{left=2.0cm,right=2.0cm,top=2.0cm,bottom=2.0cm,columnsep=0.75cm}
\usepackage{subfiles} % Use subfiles to define personal information.
\usepackage{paracol}

% % Change the font if you want to, depending on whether
% % you're using pdflatex or xelatex/lualatex
% \ifxetexorluatex
%   % If using xelatex or lualatex:
%   \setmainfont{Roboto Slab}
%   \setsansfont{Lato}
%   \renewcommand{\familydefault}{\sfdefault}
% \else
%   % If using pdflatex:
%   \usepackage[rm]{roboto}
%   \usepackage[defaultsans]{lato}
%   % \usepackage{sourcesanspro}
%   \renewcommand{\familydefault}{\sfdefault}
% \fi

% Fork (before v1.6.5a): Change the color codes to test your personal variant on any mode
\ifdarkmode%
  \definecolor{PrimaryColor}{HTML}{C69749}
  \definecolor{SecondaryColor}{HTML}{D49B54}
  \definecolor{ThirdColor}{HTML}{1877E8}
  \definecolor{BodyColor}{HTML}{ABABAB}
  \definecolor{EmphasisColor}{HTML}{ABABAB}
  \definecolor{BackgroundColor}{HTML}{191919}
\else%
%  \definecolor{PrimaryColor}{HTML}{001F5A}
%  \definecolor{SecondaryColor}{HTML}{0039AC}
%  \definecolor{ThirdColor}{HTML}{F3890B}
%  \definecolor{BodyColor}{HTML}{666666}
%  \definecolor{EmphasisColor}{HTML}{2E2E2E}
%  \definecolor{BackgroundColor}{HTML}{E2E2E2}
  \definecolor{PrimaryColor}{HTML}{001F5A}
  \definecolor{SecondaryColor}{HTML}{448888}
  \definecolor{ThirdColor}{HTML}{4682B4}
  \definecolor{BodyColor}{HTML}{666666}
  \definecolor{EmphasisColor}{HTML}{2E2E2E}
  \definecolor{BackgroundColor}{HTML}{FFFFFF}
%  \definecolor{BackgroundColor}{HTML}{E2E2E2}
\fi%

%My Colours:
% Teal 448888
% Orange DD5522 
% Light blu 77EEEE
% Steel Blu 4682B4

\colorlet{name}{PrimaryColor}
\colorlet{tagline}{SecondaryColor}
\colorlet{heading}{PrimaryColor}
\colorlet{headingrule}{ThirdColor}
\colorlet{subheading}{SecondaryColor}
\colorlet{accent}{SecondaryColor}
\colorlet{emphasis}{EmphasisColor}
\colorlet{body}{BodyColor}
\pagecolor{BackgroundColor}

%\usepackage{hyperref
%%Setup bib hyperlinks
\hypersetup{%
	urlcolor=SecondaryColor,
	citecolor=SecondaryColor
}

% Change some fonts, if necessary
\renewcommand{\namefont}{\Huge\rmfamily\bfseries}
\renewcommand{\personalinfofont}{\small\bfseries}
\renewcommand{\cvsectionfont}{\LARGE\rmfamily\bfseries}
\renewcommand{\cvsubsectionfont}{\large\bfseries}

% Change the bullets for itemize and rating marker
% for \cvskill if you want to
\renewcommand{\itemmarker}{{\small\textbullet}}
\renewcommand{\ratingmarker}{\faCircle}

\newcommand{\textalignment}{
	\justifying
    \tolerance=1 %https://tex.stackexchange.com/a/5039/137109
    \emergencystretch=\maxdimen
    \hyphenpenalty=10000 
    \hbadness=10000
}

\setlength{\parindent}{24pt}
\newcommand{\pind}{\hspace{24pt}}

\begin{document}
    
    \IfFileExists{./cl-intro.info}{
            \subfile{./cl-intro.info}
        }{
            \subfile{./cl-intro-example.info}
        }

    \vspace{1em}
    {\textalignment
    
	% We are currently seeking a Beamline Scientist – Spectroscopy who displays initiative, drive, and passion for collaboration in the interface between science and technology, to join our team on a 2-year contract basis. 
	% The Spectroscopy Group comprises four beamlines: X-ray Absorption Spectroscopy (XAS), Medium-Energy XAS-1/2 (MEX-1/2), and Soft X-ray Spectroscopy. This position focuses primarily on MEX-2, offering significant opportunities to collaborate across other beamlines in the Group and contribute meaningfully to the entire team and the broader facility.
	%  You will have the chance to collaborate with scientists, researchers, and industry partners across a wide array of disciplines, giving them unique opportunities to innovate using cutting-edge scientific technology. Reporting to the Beamline Group Manager, you will be supported to develop knowledge of the capabilities, techniques and instrumentation within the Spectroscopy Group. You will provide user support for experiments, work 	collaboratively to maintain equipment for the Beamline, and engage in outreach and research activities within the community.

	% Our team is committed to providing expert guidance, service, and promoting safe working practices. We believe that diverse teams produce better outcomes, and we value the merit a diverse perspective can bring to our team. We prioritise building inclusive communities and fostering productive relationships. We also value collaboration and effective communication, and we actively engage with the community to share the impact of our scientific achievements through compelling stories.

	% This role is ideal for you if you enjoy engaging with people, being hands-on, learning continuously, and working collaboratively. – Your insights are important to us and will contribute to developing our teams and the organisation. Together we will contribute to scientific outcomes that benefit people and the environment. 
	
	% Your main responsibilities will include, but are not limited to:
    % Provide scientific and technical support, advice and training to national and international synchrotron beamline users of the Spectroscopy Group (with focus on the MEX-2 beamline).
    % Develop knowledge of industry best practice and technology to contribute to development of the synchrotron beamline and associated facilities.
    % Work collaboratively and share scientific expertise to positively contribute to the science and research culture within the company.
    % Liaise with both internal and external stakeholders to develop a thriving synchrotron research community.
    % Develop or support outreach activities such as participating in professional forums relevant to your research interests or engaging with the broader scientific community.
    % Ensure appropriate policy, procedures, and guidelines are adhered to associated with the beamline and facility in particular in relation to WHS, radiation safety and equipment.
    % Undertake specific beamline responsibilities as assigned by the Beamline Group Manager.

	% To be successful in this position you will have: 
    % Hands-on experience with, and a good understanding of, X-ray techniques, synchrotron methods, and/or analytical instrumentation.
    % Excellent interpersonal and communication skills, and the ability to work collaboratively and share knowledge and information with colleagues, clients, and other stakeholders.
    % Demonstrated contribution to a successful and diverse team.
    % Ability to train and teach others and a strong customer focus.
    % (Experience in a user-, teaching-, or service-based environment supporting clients will be viewed favourably.)
    % Ability to undertake self-directed learning.
    % Demonstrated research contribution to scientific publications.
    % Demonstrated ability to follow policy, procedures and guidelines.

	% You must be an Australian Citizen to apply.
	% Applications will close on Wednesday 16th July 2025
	% Note, this position is also subject to a security assessment and medical check.

	% Structure:
	% 1. Who I am
	    %  a) Engaging with people, continuous learning, collaboration.
		%  b) 
	\pind I am writing to apply for the position of Beamline Scientist as advertised through\cvreference{SEEK (85363776)}{https://www.seek.com.au/job/85363776}for the Spectroscopy Group. This opportunity is of great interest to me, on account of the user facing role, multidisciplinary nature, locality in Melbourne and intersectionality of experiment, control and analysis, through all of which believe I am ideally qualified for your team. 

    % To be successful in this position you will have: 
    % A PhD in Chemistry, Physics, Earth and Environmental Science, Materials Science, Biological Science, or a related discipline.
    % (Post-doctoral experience, or equivalent, will be advantageous and viewed favourably.)
    % Hands-on experience with, and a good understanding of, X-ray techniques, synchrotron methods, and/or analytical instrumentation.
    % Demonstrated research contribution to scientific publications.
    % Ability to undertake self-directed learning.
    \pind My current role is a postdoctoral research fellow in the Materials Science Engineering department at Monash University, under the supervision of Prof. Christopher McNeill, who is an expert in using synchrotron radiation to study photovoltaic materials. I have extensive experience in spectroscopy and diffraction beamline science, and I have completed 9 beamtimes (plus discretionary time) thus far, with two more beamtimes scheduled at the end of this month. My research focus is on utilising polarized light at resonance to understand alignment and orientation in organic semiconductors (materials with some moderate disorder). I am extremely familiar with the MEX-2 beamline, which is essential to our group's research to probe the transition dipole moments of sulfur and chlorine constituents in organic semiconductors, at tender energies.
    
    \pind Complementing my postdoctoral experience, I have a PhD in condensed matter physics from Monash University under the supervision of Prof. Michael Fuhrer, researching surface science of Dirac materials and their influence to material proximity. As a probe for magnetic exchange between topological and magnetic materials, I have used neutron reflectometry (at the ACNS, ANSTO), cryogenic electronic transport and molecular beam epitaxy (in UHV). My experience as part of FLEET has also allowed me to grow a diverse research network across Australia. %Dirac surface materials such as graphene. 
    
    \pind In both of these research contexts I have demonstrated my ability to undertake self-directed learning in new fields that differ from my previous educational experiences. One such highlight is\cvreference{10.1021/ACS.NanoLett.2c03492,}{https://pubs.acs.org/doi/full/10.1021/acs.nanolett.2c03492}where I investigated the influence of a liquid-metal printed dielectric on the Dirac surface states of graphene, with collaborators at RMIT who specialise in the formation of materials at the liquid-metal surface. The emerging field of liquid-metals offer many new experimental possibilities (ex-situ and in-situ) in surface science and spectroscopy. In my postdoctoral role I am working in new collaborations across multiple existing projects with colleagues at European Synchrotrons (such as ESRF and SLS) and will be travelling in October to collaborate and present my work.

    % To be successful in this position you will have: 
    % Excellent interpersonal and communication skills, and the ability to work collaboratively and share knowledge and information with colleagues, clients, and other stakeholders.
    % Demonstrated contribution to a successful and diverse team.
    % Ability to train and teach others and a strong customer focus.
    % (Experience in a user-, teaching-, or service-based environment supporting clients will be viewed favourably.)
    \pind The culmination of my experience demonstrates excellent communication and interpersonal skills through a combination of collaboration, teaching, volunteering and community activities (as detailed in my CV). In particular, I have trained multiple PhD students in nanofabrication and measurement techniques such as molecular beam expitaxy (MBE) and lithography (EBL \& optical), supervised numerous undergraduate/masters students, and taught spectroscopic analysis processes to PhD students in my group. My leadership developed through such activities, particularly community service and governance roles, has taught me the importance of relationships with clients and stakeholders, and how to create meaningful community connections and foster positive and inclusive environments. This record demonstrates my personality to take initiative in making relational effort and contributions.

    % To be successful in this position you will have: 
    % Demonstrated ability to follow policy, procedures and guidelines.
    \pind I have a high capacity for details, particularly policy and procedures, with significant experience in preparing standard operating procedures, risk assessments, and other safety documentation required for the experimental research contexts I have utilised. The same is true for understanding compliance and best-practise in engineering and scientific research problem solving where details around specifications are critical. My diverse educational experience (including my honours degree in electrical and computer systems engineering) also positions me to effectively communicate with a wide range of technical staff and researchers to deliver ideal outcomes for the Spectroscopy Group.

    I hope you find my experiences promising and I look forward to an opportunity to discuss my application further. I am excited about the possibility of contributing to this group and the broader synchrotron community, and I believe that my skills and experience align well with the requirements of this position.
	% 2. What I have achieved [Doctoral PhD, Postdoc., Experience in Leadership roles, Public engagement & outreach]
	% 3. Why my achievements are relevant for this position
	% 4. What I can bring to the table that distinguishes my from others.
	% 		- UHV
	%		- Industry
	%		- Programming
	%		- EPICs & QANT
	% 		- Several beamlines
	%		- Postdoctoral experience in Synchrotron science.
	% 5. Why this role is important for my future.
    }

    {\color{emphasis}
    \textalignment

		
    \vspace{1em}

    Yours Sincerely, \newline

    {\color{emphasis}Matthew G Gebert}

    }
    
    % \vspace*{0.5em}
    \divider
    \IfFileExists{./cl-personal.info}{
        \vspace{-2em}
        \subfile{./cl-personal.info}
    }{
        \vspace{-2em}
        \subfile{./cl-personal-example.info}
    }

\end{document}