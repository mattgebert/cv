\documentclass[10pt,a4paper,ragged2e,withhyper,paragraphstrue]{altacv}

% \newenvironment{sloppypar*}{\sloppy\ignorespaces}{\par}
\geometry{left=1.2cm,right=1.2cm,top=1cm,bottom=1cm,columnsep=0.75cm}
\usepackage{subfiles} % Use subfiles to define personal information.
\usepackage{paracol}

% % Change the font if you want to, depending on whether
% % you're using pdflatex or xelatex/lualatex
% \ifxetexorluatex
%   % If using xelatex or lualatex:
%   \setmainfont{Roboto Slab}
%   \setsansfont{Lato}
%   \renewcommand{\familydefault}{\sfdefault}
% \else
%   % If using pdflatex:
%   \usepackage[rm]{roboto}
%   \usepackage[defaultsans]{lato}
%   % \usepackage{sourcesanspro}
%   \renewcommand{\familydefault}{\sfdefault}
% \fi

% Fork (before v1.6.5a): Change the color codes to test your personal variant on any mode
\ifdarkmode%
  \definecolor{PrimaryColor}{HTML}{C69749}
  \definecolor{SecondaryColor}{HTML}{D49B54}
  \definecolor{ThirdColor}{HTML}{1877E8}
  \definecolor{BodyColor}{HTML}{ABABAB}
  \definecolor{EmphasisColor}{HTML}{ABABAB}
  \definecolor{BackgroundColor}{HTML}{191919}
\else%
%  \definecolor{PrimaryColor}{HTML}{001F5A}
%  \definecolor{SecondaryColor}{HTML}{0039AC}
%  \definecolor{ThirdColor}{HTML}{F3890B}
%  \definecolor{BodyColor}{HTML}{666666}
%  \definecolor{EmphasisColor}{HTML}{2E2E2E}
%  \definecolor{BackgroundColor}{HTML}{E2E2E2}
  \definecolor{PrimaryColor}{HTML}{001F5A}
  \definecolor{SecondaryColor}{HTML}{448888}
  \definecolor{ThirdColor}{HTML}{4682B4}
  \definecolor{BodyColor}{HTML}{666666}
  \definecolor{EmphasisColor}{HTML}{2E2E2E}
  \definecolor{BackgroundColor}{HTML}{FFFFFF}
%  \definecolor{BackgroundColor}{HTML}{E2E2E2}
\fi%

%My Colours:
% Teal 448888
% Orange DD5522 
% Light blu 77EEEE
% Steel Blu 4682B4

\colorlet{name}{PrimaryColor}
\colorlet{tagline}{SecondaryColor}
\colorlet{heading}{PrimaryColor}
\colorlet{headingrule}{ThirdColor}
\colorlet{subheading}{SecondaryColor}
\colorlet{accent}{SecondaryColor}
\colorlet{emphasis}{EmphasisColor}
\colorlet{body}{BodyColor}
\pagecolor{BackgroundColor}

%\usepackage{hyperref
%%Setup bib hyperlinks
\hypersetup{%
	urlcolor=SecondaryColor,
	citecolor=SecondaryColor
}

% Change some fonts, if necessary
\renewcommand{\namefont}{\Huge\rmfamily\bfseries}
\renewcommand{\personalinfofont}{\small\bfseries}
\renewcommand{\cvsectionfont}{\LARGE\rmfamily\bfseries}
\renewcommand{\cvsubsectionfont}{\large\bfseries}

% Change the bullets for itemize and rating marker
% for \cvskill if you want to
\renewcommand{\itemmarker}{{\small\textbullet}}
\renewcommand{\ratingmarker}{\faCircle}

\newcommand{\textalignment}{
	\justifying
    \tolerance=1 %https://tex.stackexchange.com/a/5039/137109
    \emergencystretch=\maxdimen
    \hyphenpenalty=10000 
    \hbadness=10000
}

\setlength{\parindent}{24pt}
\newcommand{\pind}{\hspace{24pt}}

\begin{document}
    
    \IfFileExists{./cl-intro.info}{
            \subfile{./cl-intro.info}
        }{
            \subfile{./cl-intro-example.info}
        }

    \vspace{1em}
    
    %% Proposal describing how I will best suit the following key-selection criteria for the Synchrotron SXR beamline position
	
	%	Qualifications and Experience:
	%    1. PhD in a relevant field of science (e.g. physics, materials science, Earth science, biology, chemistry, etc) [Essential]. 
	%		Post-doctoral or equivalent experience in a relevant area of research. [Desirable]
	%    2.	Hands-on experience with, and a good understanding of, X-ray techniques, synchrotron methods, and/or analytical instrumentation relevant to the role, including the ability to acquire, process, analyse, and interpret data. [Essential]   
	%		Experience with vacuum science, technology, and instrumentation. [Desirable]
	%    3.	Excellent interpersonal and communication skills to work collaboratively and willingly share knowledge and information with colleagues, users and other stakeholders. [Essential]
	%    4.	Demonstrated ability to produce scientific or research outcomes. [Essential]
	%    5.	Strong customer focus [Essential]
	%		Experience in a user-, teaching, or service-based environment supporting clients or students. [Desirable]
	
	%	Professional Skills and Personal Attribute Selection Criteria (assessed at interview stage)
	%	Personal qualities that will add value to a diverse team. [Essential]
	%	•    Ability to work independently and able to plan and manage time to meet deadlines and objectives. [Essential]
	%	•    Ability to train and teach others. [Essential]
	%	•    Initiative and drive, and the ability to learn and solve problems in a dynamic environment, independently and in teams. [Essential]
	%	•    Strong organisational skills and ability to follow procedures and guidelines. [Essential]
	%	•    Good technical and instrumentation skills. Experience with programming or scientific computing. [Desirable]
	
	% Structure:
	% 1. Who I am
	% 2. What I have achieved [Doctoral PhD, Postdoc., Experience in ]
	% 3. Why my achievements are relevant for this position
	% 4. What I can bring to the table that distinguishes my from others.
	% 		- UHV
	%		- Industry
	%		- Programming
	%		- EPICs & QANT
	% 		- Several beamlines
	%		- Postdoctoral experience in Synchrotron science.
	% 5. Why this role is important for my future.

    {\color{emphasis}
    \textalignment
    \pind I am writing to you to apply for the position of Beamline Scientist\cvreference{(PD-2216)}{https://careers.ansto.gov.au/job/Melbourne-Beamline-Scientist-Soft-X-ray-Spectroscopy-VIC/1057680666/}for the Soft X-ray / Spectroscopy group. This opportunity is of significant interest to me because of its interdisciplinary possibilities and cutting-edge technological advancement, in addition to the locality of Melbourne.
    % 1. Who I am
    As of February 2024, my doctorate was conferred in experimental condensed-matter physics, supervised by Director Prof. Michael Fuhrer at the ARC Centre of Excellence in Future Low Energy Electronics and Technologies (FLEET). I also hold an honours degree in electrical engineering, majoring in digital and analogue control systems.
    In March, I started a postdoctoral research position under\cvreference{Prof. Christopher McNeill,}{https://www.monash.edu/engineering/chrismcneill}who specialises in using synchrotron techniques to understand the microstructure of organic semiconductors.
	
	% 2. What I have achieved [Doctoral PhD, Postdoc., Experience in ]
	\pind Through my honours year and doctoral research, I investigated Dirac materials such as graphene and topological insulators. These materials exhibit novel surface science, a result of their electronic Dirac surface states that arise from topology and symmetry. I have over four years of experience in the complex environments and processes that are required to grow and characterise such materials, inclusive of cyrogenic and UHV equipment for growth/measurements. Many processes were undertaken in contexts (such as the MCN) where regulated cleanliness, documenting safety procedures and protocols are paramount.
	The research I have undertaken is highly collaborative. One such highlight is\cvreference{10.1021/ACS.NanoLett.2c03492,}{https://pubs.acs.org/doi/full/10.1021/acs.nanolett.2c03492}where I investigated the influence of a liquid-metal printed dielectric on the Dirac surface states of graphene, with collaborators at RMIT who specialise in the formation of materials at the liquid-metal surface. The emerging field of liquid-metals offer many new experimental possibilities (ex-situ and in-situ) in surface science and spectroscopy. A second highlight of my doctoral research are experiments conducted at ANSTO's Lucas Heights facility, where I used the Platypus beamline to perform neutron reflectometry measurements with Dr. David Cortie, and gained familiarity with\cvreference{Refnx,}{https://github.com/refnx/refnx}a python neutron/x-ray reflectometry tool created by Dr. Andrew Nelson (ANSTO), which ignited personal interest in open-source research software engineering, with particular excellence in accessibility for users.
	Other demonstrations of my willingness to share knowledge and information include governance, outreach and teaching experiences, highlighted by a Monash Science Faculty Outstanding Student Award in 2020.
	
	\pind In my role as a postdoctoral fellow, I have gained significant relevant experience in beamline diffraction and spectroscopy measurements of organic semiconductors. We have been pushing boundaries on utilising resonance edges to resolve new  micro-structural information. 
	As part of this role and supported by the Synchrotron's ISAP program, I have developed international relationships and expertise at the 12-ID (Soft Matter Interfaces) beamline at Brookhaven National Lab's NSLS-II, performing tender-WAXS experiments. I have also conducted multiple beamtimes on each of the Synchrotron's SAXS/WAXS beamline, the MEX2 spectroscopy beamline, and the NEXAFS end-station of the SXR beamline. I have strong experience in relevant data handling, including Bayesian modelling and fitting, and the training of PhD students.
	Furthermore, part of my time as a research fellow has been invested into developing open-source python spectroscopy packages that have both programmatic and graphic interfaces. I have a strong familiarity across data formats, and have developed tools for interpreting the SXR/MEX NEXAFS data formats, in additional to performing normalisation and providing real-time data retrieval. I have also been modernising\cvreference{kkCalc}{https://github.com/mattgebert/kkcalc/tree/v2/}in correspondence with Dr. Ben Watts (PSI) for which I won best ECR poster at the Vic. X-ray Symposium 2024, which enables stitching NEXAFS spectra onto an atomic scattering database and consequent Kramers Kronig transform calculation of a dispersive spectrum. I have concurrently gained an excellent grasp on modern software-engineering principles such as the continuous integration (CI) of unit-testing, version control, typed-docs and linting, which are  essential for user-focused software and engagement.
		
	% 4. What I can bring to the table that distinguishes my from others.
	% 		- UHV
	%		- Industry
	%		- Programming
	%		- EPICs & QANT
	% 		- Several beamlines
	%		- Postdoctoral experience in Synchrotron science.
	%		- Cross field expertise between beamline science.
	\pind These aforementioned experiences demonstrate that I have a significant and diverse background that would highly advantage the spectroscopy group at the Australian Synchrotron. My digital-controls background, familiarity with EPICs and interest in scientific computing position me to be able to facilitate and contribute extensively to multidisciplinary work and decision making on spectroscopy beamlines. I have strong working knowledge of surface physics and instrumentation, and I am excited by the prospect of being in a user-facing role given my significant communication and teaching experience. I also have a strong desire to make programmatic interfacing and experimental scripting accessible (where appropriate) to users, particularly as data-driven automation and programming fluency rise. My research and industry networks developed through FLEET (including serving on an industry relations committee) and my postdoctoral work give me exposure to emerging materials technologies and research that can benefit from synchrotron science.
	
	% 5. Why this role is important for my future.
	\pind This role would be ideal for my next career step whilst allowing me to support my family, and I am very excited at the prospect of being able to contribute to the significant work that is undertaken at the Synchrotron. If you have any questions or seek further clarification, please don't hesitate to contact me. I look forward to hearing from you soon. 
%    
%    % 2. What I have achieved
%  	
%  	Additionally I've performed neutron and X-ray reflectometry techniques at ANSTO's Australian Centre for Neutron Scattering (ACNS), in collaboration with Dr. David Cortie (ACNS) and Dr. Golrokh Akhgar (CSIRO). We did this to probe the interaction between magnetic insulators and the surface states of the topological features, and have a manuscript in preparation.
%  	    
%    % 3. What I intend to do
%    \pind In my new role with Prof. McNeill, I will focus on harnessing resonant tender X-rays to understand the microstructure of organic semiconductor films. There are multiple reasons tender X-rays can be used to provide structural precision, including imaging resolution due to wavelength, relevant elements with absorption edges used in organic semiconductors, and the decreased absorption by the environment. By using the absorption edge, pre- \& post-edge contrast can be leveraged to understand the elemental contribution. However, this technique is complex in that there are many signal contributions from geometry, polarisation, diffraction, reflection e.t.c., that need to be detangled. The intention of this project is to demonstrate the successful use of technique and develop significant analysis software architecture to equip additional research. 
%    Successfully utilising tender x-rays will not only offer strong information into the packing of such polymers, but also allow liquid cell measurements of such suspended polymers, where such apparatuses in soft-Xray regimes would have suffered from absorption due to solvents - tender edges provide an exclusive probe to the polymer without solvent interference. 
%    
%    % 4. Why 5th AOFSRR is essential to what I intend to do 
%    \pind The 5\textsuperscript{th} AOFSRR School is essential for me to gain expert knowledge in a variety of synchrotron techniques and capabilities. The combination of lectures and practical sessions will fill in gaps of understanding as well as provide me general awareness of common best-practice for such research. The opportunity to extend my professional network, not only with the staff at the Australian Synchrotron, but with a group of users within Asia-Oceania will be immensely valuable. The opportunity to conduct a practical on two beamlines will also offer insight into their advanced operation, particularly ones that I need to use, but am inexperienced in. For example, I recently performed my first GI-WAXS in April, but I am not familiar alternative modes of operation for the SAXS-WAXS beamline and the implication of detector distances. MEX and SXR are also very important beamlines for my main project. I'd also be fascinated to see the infrared Microscopy (IRM) beamline.
%    
%    % 5. What I hope to contribute in the future.
%    \pind I have strong expectations that my future career pathways will be contributing to synchrotron science, either as a researcher, instrument scientist or controls/algorithmic contributor. I have already begun contributing to a new NEXAFS anaylsis repository, and will be developing further tools throughout my postdoctoral work in conjunction with beamtimes at the Australian Synchrotron. My diverse background (with a double undergraduate degree in Electrical and Computer systems, in addition to science) means I can communicate across a significant number of stakeholder groups within synchrotron research, which makes me an exceptional candidate to be part of a program like this.
%    
%    % 6. Summary why I am a valuable member to have
%    \pind As I have indicated, I believe this school is of real value to my development as I have begun to gain expertise in Synchrotron science. I have clear and long-term commitment to contributing to diverse and complex science, and all aspects of my skillset (and it's breadth) can be utilised in a synchrotron context, making me an excellent choice of candidate to award an Australian placement in the AOFSRR school. I am very excited by the opportunity to be a part of such a program. 
%    
%    \pind If you have any questions or seek further clarification, please don't hesitate to contact me. I look forward to hearing from you soon.
    
    \vspace{1em}

    Yours Sincerely, \newline

    {\color{emphasis}Matthew G Gebert}

    }
    
    % \vspace*{0.5em}
    \divider
    \IfFileExists{./cl-personal.info}{
        \vspace{-2em}
        \subfile{./cl-personal.info}
    }{
        \vspace{-2em}
        \subfile{./cl-personal-example.info}
    }

\end{document}